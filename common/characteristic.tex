
{\actuality} 

В последнее время автономная навигации мобильного агента быстро развивается и требует эффективных методов для ее решения. Навигация в данной работе 
представляет с собой планирование движения, в результате которого робот может перемещаться безопасным образом из одной позиции до другой. 
Особую роль в автономной навигации играет учет кинодинамических ограничений агента(робота). Кинодинамический учет неголомонных мобильных 
роботов представляет собой сложную задачу в планировании движения, не имеющую универсального решения и увеличивающую время работы имеющихся 
алгоритмов.

На протяжении длительного времени учет кинодинамических ограничений решался с помощью двух способов: кинодинамическое планирование
и пост-сглаживание. Алгоритмы пост-сглаживания обычно быстро решают задачу автономной навигации за счет быстрого построения
геометрической начальной догадки и затем быстрой пост-обработки траектории для учета кинодинамических ограничений робота. Однако, не все 
геометрическиеначальные догадки можно преобразовать в кинодинамическую траекторию. Для этого было предложено улучшение за счет 
дискретного сэмплирования (улучшение GRIPS). Кинодинамическое планирование проводится помощью так называемых примитивов движения, которые вместе 
с алгоритмами семейства A* и семейства RRT решали данную проблему. Однако у последних есть ограничения: они не адаптивны, что затрудняет 
планирование при изменении локальной видимости робота. Современные методы используют обучаемые методы на основе обучения с подкреплением для 
построения адаптивных примитивов движения, которые позволяют адаптироваться к изменяниям в среде.

Кроме того, чтобы учесть кинодинамические ограничения робота в последнее время стало популярным учет безопасных ограничений робота на основе
обучения с подкреплением. Алгоритмы, которые будут работать на практике должны соблюдать некоторые ограничения по безопасности. В связи с этим,
был предложен алгоритм, которые по мимо кинодинамических ограничений робот, еще учитывает безопаную составляющую.


% \ifsynopsis
% Этот абзац появляется только в~автореферате.
% Для формирования блоков, которые будут обрабатываться только в~автореферате,
% заведена проверка условия \verb!\!\verb!ifsynopsis!.
% Значение условия задаётся в~основном файле документа (\verb!synopsis.tex! для
% автореферата).
% \else
% Этот абзац появляется только в~диссертации.
% Через проверку условия \verb!\!\verb!ifsynopsis!, задаваемого в~основном файле
% документа (\verb!dissertation.tex! для диссертации), можно сделать новую
% команду, обеспечивающую появление цитаты в~диссертации, но~не~в~автореферате.
% \fi

% {\progress}
% Этот раздел должен быть отдельным структурным элементом по
% ГОСТ, но он, как правило, включается в описание актуальности
% темы. Нужен он отдельным структурынм элемементом или нет ---
% смотрите другие диссертации вашего совета, скорее всего не нужен.

{\aim} данной работы является исследование методов и разработка алгоритмов
автономной навигации мобильного агента с его учетом кинодинамических ограничений

Для~достижения поставленной цели необходимо было решить следующие {\tasks}:
\begin{enumerate}[beginpenalty=10000] % https://tex.stackexchange.com/a/476052/104425
  \item Исследовать методы и улучшить алгоритм автономной навигации на основе сглаживания траекторий GRIPS.
  \item Исследовать методы и разработать алгоритм автономной навигации на стыке основе обучения с подкреплением и классического планирования.
  \item Исследовать методы и разработать алгоритм безопасной автономной навигации на основе обучения с подкреплением.
\end{enumerate}


{\novelty}
\begin{enumerate}[beginpenalty=10000] % https://tex.stackexchange.com/a/476052/104425
  \item Преложено улучшение метода GRIPS для решения задачи автономной навигации с учетом кинодинамических ограничений робота. 
  Демонстрируется преимущество этого метода по сравнению с оригинальным алгоритом и другим методом сглаживаний траекторий.
  \item Разработан новый подход автономной навигации агента с учетом кинодинамических ограничеограниченийни для высокодинамичных сред, сочетающий 
  глобальное классическое планирование и локальное обучаемое планирование. Демонстрируется преимущество разработанного метода перед другими 
  state-of-the-art методами. 
  \item Разработан новый подход безопасной автономной навигации с учетом кинодинамических ограничений на основе безопасного и иерархического
  обучения с подкреплением. Демонстрируется преимущество разработанного метода перед другими state-of-the-art методами. \ldots
\end{enumerate}

{\influence}
\begin{enumerate}[beginpenalty=10000]
    \item Разработанные методы могут применяться в автономной навигации мобильных роботов. В городских и внутренних помещениях где требуется
    real time performance. А также в обучении и исследованиях в области
\end{enumerate}

{\methods}
Разрабатываемые алгоритмы основываются на методов теории графов, машинного обучения, дискретной оптимизации. Основным методом оценки эффективности
представленных в работе подходов численный эксперимент, которые проводится по множеству тестов в одинаковых условиях. 
Для реализации всех подходов были использованы язык программирования C++ и Python3, а также множество открытыхмодулей, предоставляющих различные 
функции – от визуализации до библиотек обучения нейронных сетей. В процессе разработки использовались библиотека глубокого обучения PyTorch, 
в качестве метода оптимизации использовался стохастический градиентный спуск адаптивной оценкой моментов Adam. 

{\defpositions}
\begin{enumerate}[beginpenalty=10000] % https://tex.stackexchange.com/a/476052/104425
  \item Улучшенный метод GRIPS на основе сглаживания траекторий, который способен работать в real time с учетом кинодинамических ограничений 
  робота.
  \item Подход POLAMP для решения задачи автономной навигации в высокодинамичной среде с динамическими препятствиями. Особенностью алгоритма 
  является работа с большим количеством динамических препятятивйи.
  \item Подход POSACON для решения задачи безопасной автономной навигации в сложных средах. Особенностью алгоритма является учет безопасного 
  поведения робота за счет объединенного подхода безопасного и иерархического обучения с подкреплением.

\end{enumerate}
В папке Documents можно ознакомиться с решением совета из Томского~ГУ
(в~файле \verb+Def_positions.pdf+), где обоснованно даются рекомендации
по~формулировкам защищаемых положений.

{\reliability} полученных результатов обеспечивается \ldots \ Результаты находятся в соответствии с результатами, полученными другими авторами.


{\probation}
Основные результаты работы докладывались~на:
перечисление основных конференций, симпозиумов и~т.\:п.

{\contribution} Автор принимал активное участие \ldots

\ifnumequal{\value{bibliosel}}{0}
{%%% Встроенная реализация с загрузкой файла через движок bibtex8. (При желании, внутри можно использовать обычные ссылки, наподобие `\cite{vakbib1,vakbib2}`).
    {\publications} Основные результаты по теме диссертации изложены
    в~XX~печатных изданиях,
    X из которых изданы в журналах, рекомендованных ВАК,
    X "--- в тезисах докладов.
}%
{%%% Реализация пакетом biblatex через движок biber
    \begin{refsection}[bl-author, bl-registered]
        % Это refsection=1.
        % Процитированные здесь работы:
        %  * подсчитываются, для автоматического составления фразы "Основные результаты ..."
        %  * попадают в авторскую библиографию, при usefootcite==0 и стиле `\insertbiblioauthor` или `\insertbiblioauthorgrouped`
        %  * нумеруются там в зависимости от порядка команд `\printbibliography` в этом разделе.
        %  * при использовании `\insertbiblioauthorgrouped`, порядок команд `\printbibliography` в нём должен быть тем же (см. biblio/biblatex.tex)
        %
        % Невидимый библиографический список для подсчёта количества публикаций:
        \printbibliography[heading=nobibheading, section=1, env=countauthorvak,          keyword=biblioauthorvak]%
        \printbibliography[heading=nobibheading, section=1, env=countauthorwos,          keyword=biblioauthorwos]%
        \printbibliography[heading=nobibheading, section=1, env=countauthorscopus,       keyword=biblioauthorscopus]%
        \printbibliography[heading=nobibheading, section=1, env=countauthorconf,         keyword=biblioauthorconf]%
        \printbibliography[heading=nobibheading, section=1, env=countauthorother,        keyword=biblioauthorother]%
        \printbibliography[heading=nobibheading, section=1, env=countregistered,         keyword=biblioregistered]%
        \printbibliography[heading=nobibheading, section=1, env=countauthorpatent,       keyword=biblioauthorpatent]%
        \printbibliography[heading=nobibheading, section=1, env=countauthorprogram,      keyword=biblioauthorprogram]%
        \printbibliography[heading=nobibheading, section=1, env=countauthor,             keyword=biblioauthor]%
        \printbibliography[heading=nobibheading, section=1, env=countauthorvakscopuswos, filter=vakscopuswos]%
        \printbibliography[heading=nobibheading, section=1, env=countauthorscopuswos,    filter=scopuswos]%
        %
        \nocite{*}%
        %
        {\publications} Основные результаты по теме диссертации изложены в~\arabic{citeauthor}~печатных изданиях,
        \arabic{citeauthorvak} из которых изданы в журналах, рекомендованных ВАК\sloppy%
        \ifnum \value{citeauthorscopuswos}>0%
            , \arabic{citeauthorscopuswos} "--- в~периодических научных журналах, индексируемых Web of~Science и Scopus\sloppy%
        \fi%
        \ifnum \value{citeauthorconf}>0%
            , \arabic{citeauthorconf} "--- в~тезисах докладов.
        \else%
            .
        \fi%
        \ifnum \value{citeregistered}=1%
            \ifnum \value{citeauthorpatent}=1%
                Зарегистрирован \arabic{citeauthorpatent} патент.
            \fi%
            \ifnum \value{citeauthorprogram}=1%
                Зарегистрирована \arabic{citeauthorprogram} программа для ЭВМ.
            \fi%
        \fi%
        \ifnum \value{citeregistered}>1%
            Зарегистрированы\ %
            \ifnum \value{citeauthorpatent}>0%
            \formbytotal{citeauthorpatent}{патент}{}{а}{}\sloppy%
            \ifnum \value{citeauthorprogram}=0 . \else \ и~\fi%
            \fi%
            \ifnum \value{citeauthorprogram}>0%
            \formbytotal{citeauthorprogram}{программ}{а}{ы}{} для ЭВМ.
            \fi%
        \fi%
        % К публикациям, в которых излагаются основные научные результаты диссертации на соискание учёной
        % степени, в рецензируемых изданиях приравниваются патенты на изобретения, патенты (свидетельства) на
        % полезную модель, патенты на промышленный образец, патенты на селекционные достижения, свидетельства
        % на программу для электронных вычислительных машин, базу данных, топологию интегральных микросхем,
        % зарегистрированные в установленном порядке.(в ред. Постановления Правительства РФ от 21.04.2016 N 335)
    \end{refsection}%
    \begin{refsection}[bl-author, bl-registered]
        % Это refsection=2.
        % Процитированные здесь работы:
        %  * попадают в авторскую библиографию, при usefootcite==0 и стиле `\insertbiblioauthorimportant`.
        %  * ни на что не влияют в противном случае
        \nocite{vakbib2}%vak
        \nocite{patbib1}%patent
        \nocite{progbib1}%program
        \nocite{bib1}%other
        \nocite{confbib1}%conf
    \end{refsection}%
        %
        % Всё, что вне этих двух refsection, это refsection=0,
        %  * для диссертации - это нормальные ссылки, попадающие в обычную библиографию
        %  * для автореферата:
        %     * при usefootcite==0, ссылка корректно сработает только для источника из `external.bib`. Для своих работ --- напечатает "[0]" (и даже Warning не вылезет).
        %     * при usefootcite==1, ссылка сработает нормально. В авторской библиографии будут только процитированные в refsection=0 работы.
}

При использовании пакета \verb!biblatex! будут подсчитаны все работы, добавленные
в файл \verb!biblio/author.bib!. Для правильного подсчёта работ в~различных
системах цитирования требуется использовать поля:
\begin{itemize}
        \item \texttt{authorvak} если публикация индексирована ВАК,
        \item \texttt{authorscopus} если публикация индексирована Scopus,
        \item \texttt{authorwos} если публикация индексирована Web of Science,
        \item \texttt{authorconf} для докладов конференций,
        \item \texttt{authorpatent} для патентов,
        \item \texttt{authorprogram} для зарегистрированных программ для ЭВМ,
        \item \texttt{authorother} для других публикаций.
\end{itemize}
Для подсчёта используются счётчики:
\begin{itemize}
        \item \texttt{citeauthorvak} для работ, индексируемых ВАК,
        \item \texttt{citeauthorscopus} для работ, индексируемых Scopus,
        \item \texttt{citeauthorwos} для работ, индексируемых Web of Science,
        \item \texttt{citeauthorvakscopuswos} для работ, индексируемых одной из трёх баз,
        \item \texttt{citeauthorscopuswos} для работ, индексируемых Scopus или Web of~Science,
        \item \texttt{citeauthorconf} для докладов на конференциях,
        \item \texttt{citeauthorother} для остальных работ,
        \item \texttt{citeauthorpatent} для патентов,
        \item \texttt{citeauthorprogram} для зарегистрированных программ для ЭВМ,
        \item \texttt{citeauthor} для суммарного количества работ.
\end{itemize}
% Счётчик \texttt{citeexternal} используется для подсчёта процитированных публикаций;
% \texttt{citeregistered} "--- для подсчёта суммарного количества патентов и программ для ЭВМ.

Для добавления в список публикаций автора работ, которые не были процитированы в
автореферате, требуется их~перечислить с использованием команды \verb!\nocite! в
\verb!Synopsis/content.tex!.
